% Options for packages loaded elsewhere
\PassOptionsToPackage{unicode}{hyperref}
\PassOptionsToPackage{hyphens}{url}
\PassOptionsToPackage{dvipsnames,svgnames,x11names}{xcolor}
%
\documentclass[
  letterpaper,
  DIV=11,
  numbers=noendperiod]{scrartcl}

\usepackage{amsmath,amssymb}
\usepackage{iftex}
\ifPDFTeX
  \usepackage[T1]{fontenc}
  \usepackage[utf8]{inputenc}
  \usepackage{textcomp} % provide euro and other symbols
\else % if luatex or xetex
  \usepackage{unicode-math}
  \defaultfontfeatures{Scale=MatchLowercase}
  \defaultfontfeatures[\rmfamily]{Ligatures=TeX,Scale=1}
\fi
\usepackage{lmodern}
\ifPDFTeX\else  
    % xetex/luatex font selection
\fi
% Use upquote if available, for straight quotes in verbatim environments
\IfFileExists{upquote.sty}{\usepackage{upquote}}{}
\IfFileExists{microtype.sty}{% use microtype if available
  \usepackage[]{microtype}
  \UseMicrotypeSet[protrusion]{basicmath} % disable protrusion for tt fonts
}{}
\makeatletter
\@ifundefined{KOMAClassName}{% if non-KOMA class
  \IfFileExists{parskip.sty}{%
    \usepackage{parskip}
  }{% else
    \setlength{\parindent}{0pt}
    \setlength{\parskip}{6pt plus 2pt minus 1pt}}
}{% if KOMA class
  \KOMAoptions{parskip=half}}
\makeatother
\usepackage{xcolor}
\setlength{\emergencystretch}{3em} % prevent overfull lines
\setcounter{secnumdepth}{5}
% Make \paragraph and \subparagraph free-standing
\ifx\paragraph\undefined\else
  \let\oldparagraph\paragraph
  \renewcommand{\paragraph}[1]{\oldparagraph{#1}\mbox{}}
\fi
\ifx\subparagraph\undefined\else
  \let\oldsubparagraph\subparagraph
  \renewcommand{\subparagraph}[1]{\oldsubparagraph{#1}\mbox{}}
\fi

\usepackage{color}
\usepackage{fancyvrb}
\newcommand{\VerbBar}{|}
\newcommand{\VERB}{\Verb[commandchars=\\\{\}]}
\DefineVerbatimEnvironment{Highlighting}{Verbatim}{commandchars=\\\{\}}
% Add ',fontsize=\small' for more characters per line
\usepackage{framed}
\definecolor{shadecolor}{RGB}{241,243,245}
\newenvironment{Shaded}{\begin{snugshade}}{\end{snugshade}}
\newcommand{\AlertTok}[1]{\textcolor[rgb]{0.68,0.00,0.00}{#1}}
\newcommand{\AnnotationTok}[1]{\textcolor[rgb]{0.37,0.37,0.37}{#1}}
\newcommand{\AttributeTok}[1]{\textcolor[rgb]{0.40,0.45,0.13}{#1}}
\newcommand{\BaseNTok}[1]{\textcolor[rgb]{0.68,0.00,0.00}{#1}}
\newcommand{\BuiltInTok}[1]{\textcolor[rgb]{0.00,0.23,0.31}{#1}}
\newcommand{\CharTok}[1]{\textcolor[rgb]{0.13,0.47,0.30}{#1}}
\newcommand{\CommentTok}[1]{\textcolor[rgb]{0.37,0.37,0.37}{#1}}
\newcommand{\CommentVarTok}[1]{\textcolor[rgb]{0.37,0.37,0.37}{\textit{#1}}}
\newcommand{\ConstantTok}[1]{\textcolor[rgb]{0.56,0.35,0.01}{#1}}
\newcommand{\ControlFlowTok}[1]{\textcolor[rgb]{0.00,0.23,0.31}{#1}}
\newcommand{\DataTypeTok}[1]{\textcolor[rgb]{0.68,0.00,0.00}{#1}}
\newcommand{\DecValTok}[1]{\textcolor[rgb]{0.68,0.00,0.00}{#1}}
\newcommand{\DocumentationTok}[1]{\textcolor[rgb]{0.37,0.37,0.37}{\textit{#1}}}
\newcommand{\ErrorTok}[1]{\textcolor[rgb]{0.68,0.00,0.00}{#1}}
\newcommand{\ExtensionTok}[1]{\textcolor[rgb]{0.00,0.23,0.31}{#1}}
\newcommand{\FloatTok}[1]{\textcolor[rgb]{0.68,0.00,0.00}{#1}}
\newcommand{\FunctionTok}[1]{\textcolor[rgb]{0.28,0.35,0.67}{#1}}
\newcommand{\ImportTok}[1]{\textcolor[rgb]{0.00,0.46,0.62}{#1}}
\newcommand{\InformationTok}[1]{\textcolor[rgb]{0.37,0.37,0.37}{#1}}
\newcommand{\KeywordTok}[1]{\textcolor[rgb]{0.00,0.23,0.31}{#1}}
\newcommand{\NormalTok}[1]{\textcolor[rgb]{0.00,0.23,0.31}{#1}}
\newcommand{\OperatorTok}[1]{\textcolor[rgb]{0.37,0.37,0.37}{#1}}
\newcommand{\OtherTok}[1]{\textcolor[rgb]{0.00,0.23,0.31}{#1}}
\newcommand{\PreprocessorTok}[1]{\textcolor[rgb]{0.68,0.00,0.00}{#1}}
\newcommand{\RegionMarkerTok}[1]{\textcolor[rgb]{0.00,0.23,0.31}{#1}}
\newcommand{\SpecialCharTok}[1]{\textcolor[rgb]{0.37,0.37,0.37}{#1}}
\newcommand{\SpecialStringTok}[1]{\textcolor[rgb]{0.13,0.47,0.30}{#1}}
\newcommand{\StringTok}[1]{\textcolor[rgb]{0.13,0.47,0.30}{#1}}
\newcommand{\VariableTok}[1]{\textcolor[rgb]{0.07,0.07,0.07}{#1}}
\newcommand{\VerbatimStringTok}[1]{\textcolor[rgb]{0.13,0.47,0.30}{#1}}
\newcommand{\WarningTok}[1]{\textcolor[rgb]{0.37,0.37,0.37}{\textit{#1}}}

\providecommand{\tightlist}{%
  \setlength{\itemsep}{0pt}\setlength{\parskip}{0pt}}\usepackage{longtable,booktabs,array}
\usepackage{calc} % for calculating minipage widths
% Correct order of tables after \paragraph or \subparagraph
\usepackage{etoolbox}
\makeatletter
\patchcmd\longtable{\par}{\if@noskipsec\mbox{}\fi\par}{}{}
\makeatother
% Allow footnotes in longtable head/foot
\IfFileExists{footnotehyper.sty}{\usepackage{footnotehyper}}{\usepackage{footnote}}
\makesavenoteenv{longtable}
\usepackage{graphicx}
\makeatletter
\def\maxwidth{\ifdim\Gin@nat@width>\linewidth\linewidth\else\Gin@nat@width\fi}
\def\maxheight{\ifdim\Gin@nat@height>\textheight\textheight\else\Gin@nat@height\fi}
\makeatother
% Scale images if necessary, so that they will not overflow the page
% margins by default, and it is still possible to overwrite the defaults
% using explicit options in \includegraphics[width, height, ...]{}
\setkeys{Gin}{width=\maxwidth,height=\maxheight,keepaspectratio}
% Set default figure placement to htbp
\makeatletter
\def\fps@figure{htbp}
\makeatother

\KOMAoption{captions}{tableheading}
\makeatletter
\makeatother
\makeatletter
\makeatother
\makeatletter
\@ifpackageloaded{caption}{}{\usepackage{caption}}
\AtBeginDocument{%
\ifdefined\contentsname
  \renewcommand*\contentsname{Table of contents}
\else
  \newcommand\contentsname{Table of contents}
\fi
\ifdefined\listfigurename
  \renewcommand*\listfigurename{List of Figures}
\else
  \newcommand\listfigurename{List of Figures}
\fi
\ifdefined\listtablename
  \renewcommand*\listtablename{List of Tables}
\else
  \newcommand\listtablename{List of Tables}
\fi
\ifdefined\figurename
  \renewcommand*\figurename{Figure}
\else
  \newcommand\figurename{Figure}
\fi
\ifdefined\tablename
  \renewcommand*\tablename{Table}
\else
  \newcommand\tablename{Table}
\fi
}
\@ifpackageloaded{float}{}{\usepackage{float}}
\floatstyle{ruled}
\@ifundefined{c@chapter}{\newfloat{codelisting}{h}{lop}}{\newfloat{codelisting}{h}{lop}[chapter]}
\floatname{codelisting}{Listing}
\newcommand*\listoflistings{\listof{codelisting}{List of Listings}}
\makeatother
\makeatletter
\@ifpackageloaded{caption}{}{\usepackage{caption}}
\@ifpackageloaded{subcaption}{}{\usepackage{subcaption}}
\makeatother
\makeatletter
\@ifpackageloaded{tcolorbox}{}{\usepackage[skins,breakable]{tcolorbox}}
\makeatother
\makeatletter
\@ifundefined{shadecolor}{\definecolor{shadecolor}{rgb}{.97, .97, .97}}
\makeatother
\makeatletter
\makeatother
\makeatletter
\makeatother
\ifLuaTeX
  \usepackage{selnolig}  % disable illegal ligatures
\fi
\IfFileExists{bookmark.sty}{\usepackage{bookmark}}{\usepackage{hyperref}}
\IfFileExists{xurl.sty}{\usepackage{xurl}}{} % add URL line breaks if available
\urlstyle{same} % disable monospaced font for URLs
\hypersetup{
  pdftitle={Datenmanagement und deskriptive Statistik},
  colorlinks=true,
  linkcolor={blue},
  filecolor={Maroon},
  citecolor={Blue},
  urlcolor={Blue},
  pdfcreator={LaTeX via pandoc}}

\title{Datenmanagement und deskriptive Statistik}
\author{}
\date{}

\begin{document}
\maketitle
\ifdefined\Shaded\renewenvironment{Shaded}{\begin{tcolorbox}[borderline west={3pt}{0pt}{shadecolor}, interior hidden, breakable, boxrule=0pt, frame hidden, enhanced, sharp corners]}{\end{tcolorbox}}\fi

\renewcommand*\contentsname{Table of contents}
{
\hypersetup{linkcolor=}
\setcounter{tocdepth}{3}
\tableofcontents
}
\hypertarget{sec-pot}{%
\subsection{Beispieldatensatz potato.xlsx}\label{sec-pot}}

Diese Daten sind die leicht modifizierten und erweiterten
\texttt{greenhouse}-Daten aus dem R-Paket \texttt{agricolae}.

In einem Versuch wurden verschiedene Kartoffelsorten (\texttt{variety})
in verschiedenen Anbaumethoden (\texttt{method}) kultiviert und die
Anzahl Knollen (\texttt{tubers}), deren Gewicht (\texttt{weight}) und
der Krankheitsbefall (\texttt{infection} 1=ja, 0=nein) ermittelt.

Zunächst lesen wir den Datensatz \texttt{potato.xlsx} mit der Funktion
\texttt{read.xlsx} aus der \texttt{library(openxlsx)} ein und benennen
den \texttt{data.frame} mit \texttt{pot} und machen uns mit ihm
vertraut.

\begin{Shaded}
\begin{Highlighting}[]
\FunctionTok{library}\NormalTok{(openxlsx)}
\NormalTok{pot}\OtherTok{\textless{}{-}}\FunctionTok{read.xlsx}\NormalTok{(}\StringTok{"potato.xlsx"}\NormalTok{, }\AttributeTok{sheet=}\DecValTok{1}\NormalTok{)}
\FunctionTok{str}\NormalTok{(pot)}
\end{Highlighting}
\end{Shaded}

\begin{verbatim}
'data.frame':   478 obs. of  6 variables:
 $ variety  : chr  "Unica" "Unica" "Unica" "Unica" ...
 $ method   : chr  "pot" "pot" "pot" "pot" ...
 $ plant    : num  1 2 3 4 5 6 7 8 9 10 ...
 $ tubers   : num  9 3 4 4 2 2 3 6 9 6 ...
 $ weight   : num  209.9 248.4 53.6 77.2 11.3 ...
 $ infection: num  0 0 0 0 1 0 0 0 0 0 ...
\end{verbatim}

Der Datensatz hat 478 Zeilen (Beobachtungen) und 6 Spalten.

\texttt{variety}, \texttt{method} und \texttt{infection} sind als
\emph{character} eingelesen.

\begin{Shaded}
\begin{Highlighting}[]
\FunctionTok{unique}\NormalTok{(pot}\SpecialCharTok{$}\NormalTok{variety)}
\end{Highlighting}
\end{Shaded}

\begin{verbatim}
[1] "Unica"     "Mariva"    "Costanera"
\end{verbatim}

\begin{Shaded}
\begin{Highlighting}[]
\FunctionTok{unique}\NormalTok{(pot}\SpecialCharTok{$}\NormalTok{method)}
\end{Highlighting}
\end{Shaded}

\begin{verbatim}
[1] "pot"        "bed"        "hydroponic" "aeroponic" 
\end{verbatim}

Diese können wir auch in Faktoren umwandeln (siehe
Section~\ref{sec-mutate}).

\hypertarget{datenmanagement-mit-dplyr}{%
\subsection{Datenmanagement mit dplyr}\label{datenmanagement-mit-dplyr}}

Eine gute Hilfestellung findet ihr unter:\\
\url{https://cran.r-project.org/web/packages/dplyr/vignettes/dplyr.html}

Wir laden zunächst das Paket \texttt{dplyr} ein.

\begin{Shaded}
\begin{Highlighting}[]
\FunctionTok{library}\NormalTok{(dplyr)}
\end{Highlighting}
\end{Shaded}

Um den Output der folgenden Beispiele übersichtlicher zu gestalten,
erstelle ich ein Subset aus dem Datensatz \texttt{pot}, der nur 24
Beobachtungen enthält (zwei Beobachtungen für jede Kombination aus Sorte
und Methode).

\begin{Shaded}
\begin{Highlighting}[]
\FunctionTok{set.seed}\NormalTok{(}\DecValTok{123}\NormalTok{)}
\NormalTok{pot.sub}\OtherTok{=}\NormalTok{pot}\SpecialCharTok{\%\textgreater{}\%} 
    \FunctionTok{group\_by}\NormalTok{(variety, method) }\SpecialCharTok{\%\textgreater{}\%}
    \FunctionTok{slice\_sample}\NormalTok{(}\AttributeTok{n =} \DecValTok{2}\NormalTok{)}
\FunctionTok{print}\NormalTok{(pot.sub, }\AttributeTok{n=}\DecValTok{24}\NormalTok{)}
\end{Highlighting}
\end{Shaded}

\begin{verbatim}
# A tibble: 24 x 6
# Groups:   variety, method [12]
   variety   method     plant tubers weight infection
   <chr>     <chr>      <dbl>  <dbl>  <dbl>     <dbl>
 1 Costanera aeroponic      1      5  24.1          0
 2 Costanera aeroponic      5      7  42            0
 3 Costanera bed            4      7 137.           1
 4 Costanera bed            3      5  78            1
 5 Costanera hydroponic     8      6  31.8          1
 6 Costanera hydroponic     4      5  42            1
 7 Costanera pot            5      3 110.           0
 8 Costanera pot            6      7 139.           0
 9 Mariva    aeroponic      7      6  99.2          0
10 Mariva    aeroponic      5      5  72.6          0
11 Mariva    bed            7      5  59.9          0
12 Mariva    bed            8      5  77.3          0
13 Mariva    hydroponic     9      2  10.8          0
14 Mariva    hydroponic     9      1   7.85         0
15 Mariva    pot            5      7 104.           0
16 Mariva    pot            8      5  64.8          0
17 Unica     aeroponic      6      2 112.           0
18 Unica     aeroponic      7      4  50.1          0
19 Unica     bed            9      4 124.           0
20 Unica     bed            9      3  38            0
21 Unica     hydroponic     6      2  17.1          0
22 Unica     hydroponic     4      2  21.8          0
23 Unica     pot            7      4  50.4          0
24 Unica     pot            9      5  97.7          1
\end{verbatim}

\hypertarget{zeilen}{%
\subsubsection{Zeilen}\label{zeilen}}

Wir lernen nun Funtionen aus dem paket \texttt{dplyr} kennen, mit denen
wir Beobachtungen, d.h. Zeilen unseres Datensatzes, auswählen oder
umsortieren können.

\hypertarget{filter}{%
\paragraph{filter}\label{filter}}

Mit der Funktion \texttt{filter} können wir ein Subset des Datensatzes
erstellen. \texttt{pot.M} enthält mit folgendem Code nur noch
Beobachtungen der Sorte \texttt{Mariva}.

\begin{Shaded}
\begin{Highlighting}[]
\NormalTok{pot.M}\OtherTok{\textless{}{-}}\NormalTok{ pot.sub }\SpecialCharTok{\%\textgreater{}\%} \FunctionTok{filter}\NormalTok{(variety}\SpecialCharTok{==}\StringTok{"Mariva"}\NormalTok{)}
\NormalTok{pot.M}
\end{Highlighting}
\end{Shaded}

\begin{verbatim}
# A tibble: 8 x 6
# Groups:   variety, method [4]
  variety method     plant tubers weight infection
  <chr>   <chr>      <dbl>  <dbl>  <dbl>     <dbl>
1 Mariva  aeroponic      7      6  99.2          0
2 Mariva  aeroponic      5      5  72.6          0
3 Mariva  bed            7      5  59.9          0
4 Mariva  bed            8      5  77.3          0
5 Mariva  hydroponic     9      2  10.8          0
6 Mariva  hydroponic     9      1   7.85         0
7 Mariva  pot            5      7 104.           0
8 Mariva  pot            8      5  64.8          0
\end{verbatim}

\texttt{pot.BM} enthält mit folgendem Code nur noch Beobachtungen der
Sorte \texttt{Mariva} und Methode \texttt{bed}.

\begin{Shaded}
\begin{Highlighting}[]
\NormalTok{pot.MB}\OtherTok{\textless{}{-}}\NormalTok{ pot.sub }\SpecialCharTok{\%\textgreater{}\%} \FunctionTok{filter}\NormalTok{(variety}\SpecialCharTok{==}\StringTok{"Mariva"}\NormalTok{, method}\SpecialCharTok{==}\StringTok{"bed"}\NormalTok{)}
\NormalTok{pot.MB  }
\end{Highlighting}
\end{Shaded}

\begin{verbatim}
# A tibble: 2 x 6
# Groups:   variety, method [1]
  variety method plant tubers weight infection
  <chr>   <chr>  <dbl>  <dbl>  <dbl>     <dbl>
1 Mariva  bed        7      5   59.9         0
2 Mariva  bed        8      5   77.3         0
\end{verbatim}

Mehrere Ausprägungen einer Variable können mit \texttt{\%in\%}
ausgewählt werden: \texttt{pot.MU} enthält mit folgendem Code nur noch
Beobachtungen der Sorten \texttt{Mariva} und \texttt{Unica}.

\begin{Shaded}
\begin{Highlighting}[]
\NormalTok{pot.MU}\OtherTok{\textless{}{-}}\NormalTok{ pot.sub }\SpecialCharTok{\%\textgreater{}\%} \FunctionTok{filter}\NormalTok{(variety}\SpecialCharTok{\%in\%}\FunctionTok{c}\NormalTok{(}\StringTok{"Mariva"}\NormalTok{, }\StringTok{"Unica"}\NormalTok{))}
\NormalTok{pot.MU}
\end{Highlighting}
\end{Shaded}

\begin{verbatim}
# A tibble: 16 x 6
# Groups:   variety, method [8]
   variety method     plant tubers weight infection
   <chr>   <chr>      <dbl>  <dbl>  <dbl>     <dbl>
 1 Mariva  aeroponic      7      6  99.2          0
 2 Mariva  aeroponic      5      5  72.6          0
 3 Mariva  bed            7      5  59.9          0
 4 Mariva  bed            8      5  77.3          0
 5 Mariva  hydroponic     9      2  10.8          0
 6 Mariva  hydroponic     9      1   7.85         0
 7 Mariva  pot            5      7 104.           0
 8 Mariva  pot            8      5  64.8          0
 9 Unica   aeroponic      6      2 112.           0
10 Unica   aeroponic      7      4  50.1          0
11 Unica   bed            9      4 124.           0
12 Unica   bed            9      3  38            0
13 Unica   hydroponic     6      2  17.1          0
14 Unica   hydroponic     4      2  21.8          0
15 Unica   pot            7      4  50.4          0
16 Unica   pot            9      5  97.7          1
\end{verbatim}

Alternativ können alle Beobachtungen bis auf Sorte \texttt{Mariva}
selektiert werden.

\begin{Shaded}
\begin{Highlighting}[]
\NormalTok{pot.oM}\OtherTok{\textless{}{-}}\NormalTok{ pot.sub }\SpecialCharTok{\%\textgreater{}\%} \FunctionTok{filter}\NormalTok{(variety}\SpecialCharTok{!=}\NormalTok{(}\StringTok{"Mariva"}\NormalTok{))}
\NormalTok{pot.oM }
\end{Highlighting}
\end{Shaded}

\begin{verbatim}
# A tibble: 16 x 6
# Groups:   variety, method [8]
   variety   method     plant tubers weight infection
   <chr>     <chr>      <dbl>  <dbl>  <dbl>     <dbl>
 1 Costanera aeroponic      1      5   24.1         0
 2 Costanera aeroponic      5      7   42           0
 3 Costanera bed            4      7  137.          1
 4 Costanera bed            3      5   78           1
 5 Costanera hydroponic     8      6   31.8         1
 6 Costanera hydroponic     4      5   42           1
 7 Costanera pot            5      3  110.          0
 8 Costanera pot            6      7  139.          0
 9 Unica     aeroponic      6      2  112.          0
10 Unica     aeroponic      7      4   50.1         0
11 Unica     bed            9      4  124.          0
12 Unica     bed            9      3   38           0
13 Unica     hydroponic     6      2   17.1         0
14 Unica     hydroponic     4      2   21.8         0
15 Unica     pot            7      4   50.4         0
16 Unica     pot            9      5   97.7         1
\end{verbatim}

Hier werden alle Beobachtungen mit größer gleich 5 Knollen selektiert.

\begin{Shaded}
\begin{Highlighting}[]
\NormalTok{pot.T5 }\OtherTok{\textless{}{-}}\NormalTok{pot.sub }\SpecialCharTok{\%\textgreater{}\%} \FunctionTok{filter}\NormalTok{(tubers}\SpecialCharTok{\textgreater{}=}\DecValTok{5}\NormalTok{)}
\NormalTok{pot.T5}
\end{Highlighting}
\end{Shaded}

\begin{verbatim}
# A tibble: 14 x 6
# Groups:   variety, method [8]
   variety   method     plant tubers weight infection
   <chr>     <chr>      <dbl>  <dbl>  <dbl>     <dbl>
 1 Costanera aeroponic      1      5   24.1         0
 2 Costanera aeroponic      5      7   42           0
 3 Costanera bed            4      7  137.          1
 4 Costanera bed            3      5   78           1
 5 Costanera hydroponic     8      6   31.8         1
 6 Costanera hydroponic     4      5   42           1
 7 Costanera pot            6      7  139.          0
 8 Mariva    aeroponic      7      6   99.2         0
 9 Mariva    aeroponic      5      5   72.6         0
10 Mariva    bed            7      5   59.9         0
11 Mariva    bed            8      5   77.3         0
12 Mariva    pot            5      7  104.          0
13 Mariva    pot            8      5   64.8         0
14 Unica     pot            9      5   97.7         1
\end{verbatim}

\begin{Shaded}
\begin{Highlighting}[]
\NormalTok{pot.T5}\SpecialCharTok{$}\NormalTok{tubers}
\end{Highlighting}
\end{Shaded}

\begin{verbatim}
 [1] 5 7 7 5 6 5 7 6 5 5 5 7 5 5
\end{verbatim}

Hier werden alle Beobachtungen mit mehr als 5 Knollen selektiert.

\begin{Shaded}
\begin{Highlighting}[]
\NormalTok{pot.T5 }\OtherTok{\textless{}{-}}\NormalTok{pot.sub }\SpecialCharTok{\%\textgreater{}\%} \FunctionTok{filter}\NormalTok{(tubers}\SpecialCharTok{\textgreater{}}\DecValTok{5}\NormalTok{)}
\NormalTok{pot.T5}
\end{Highlighting}
\end{Shaded}

\begin{verbatim}
# A tibble: 6 x 6
# Groups:   variety, method [6]
  variety   method     plant tubers weight infection
  <chr>     <chr>      <dbl>  <dbl>  <dbl>     <dbl>
1 Costanera aeroponic      5      7   42           0
2 Costanera bed            4      7  137.          1
3 Costanera hydroponic     8      6   31.8         1
4 Costanera pot            6      7  139.          0
5 Mariva    aeroponic      7      6   99.2         0
6 Mariva    pot            5      7  104.          0
\end{verbatim}

\begin{Shaded}
\begin{Highlighting}[]
\NormalTok{pot.T5}\SpecialCharTok{$}\NormalTok{tubers}
\end{Highlighting}
\end{Shaded}

\begin{verbatim}
[1] 7 7 6 7 6 7
\end{verbatim}

zwei Bedingungen (Anzahl Knollen und Gewicht)

\begin{Shaded}
\begin{Highlighting}[]
\NormalTok{pot.sub }\SpecialCharTok{\%\textgreater{}\%} \FunctionTok{filter}\NormalTok{(tubers}\SpecialCharTok{\textgreater{}=}\DecValTok{5}\NormalTok{, weight }\SpecialCharTok{\textgreater{}} \DecValTok{80}\NormalTok{)}
\end{Highlighting}
\end{Shaded}

\begin{verbatim}
# A tibble: 5 x 6
# Groups:   variety, method [5]
  variety   method    plant tubers weight infection
  <chr>     <chr>     <dbl>  <dbl>  <dbl>     <dbl>
1 Costanera bed           4      7  137.          1
2 Costanera pot           6      7  139.          0
3 Mariva    aeroponic     7      6   99.2         0
4 Mariva    pot           5      7  104.          0
5 Unica     pot           9      5   97.7         1
\end{verbatim}

\hypertarget{arrange}{%
\paragraph{arrange}\label{arrange}}

Mit der Funktion \texttt{arrange} sortieren wir die Daten. Hier nach
Anzahl \texttt{tubers} in aufsteigender Reihenfolge.

\begin{Shaded}
\begin{Highlighting}[]
\NormalTok{pot.sub }\SpecialCharTok{\%\textgreater{}\%} \FunctionTok{arrange}\NormalTok{(tubers)}
\end{Highlighting}
\end{Shaded}

\begin{verbatim}
# A tibble: 24 x 6
# Groups:   variety, method [12]
   variety   method     plant tubers weight infection
   <chr>     <chr>      <dbl>  <dbl>  <dbl>     <dbl>
 1 Mariva    hydroponic     9      1   7.85         0
 2 Mariva    hydroponic     9      2  10.8          0
 3 Unica     aeroponic      6      2 112.           0
 4 Unica     hydroponic     6      2  17.1          0
 5 Unica     hydroponic     4      2  21.8          0
 6 Costanera pot            5      3 110.           0
 7 Unica     bed            9      3  38            0
 8 Unica     aeroponic      7      4  50.1          0
 9 Unica     bed            9      4 124.           0
10 Unica     pot            7      4  50.4          0
# i 14 more rows
\end{verbatim}

und hier in absteigender Reihenfolge durch Funktion \texttt{desc()}:

\begin{Shaded}
\begin{Highlighting}[]
\NormalTok{pot.sub }\SpecialCharTok{\%\textgreater{}\%} \FunctionTok{arrange}\NormalTok{(}\FunctionTok{desc}\NormalTok{(tubers))}
\end{Highlighting}
\end{Shaded}

\begin{verbatim}
# A tibble: 24 x 6
# Groups:   variety, method [12]
   variety   method     plant tubers weight infection
   <chr>     <chr>      <dbl>  <dbl>  <dbl>     <dbl>
 1 Costanera aeroponic      5      7   42           0
 2 Costanera bed            4      7  137.          1
 3 Costanera pot            6      7  139.          0
 4 Mariva    pot            5      7  104.          0
 5 Costanera hydroponic     8      6   31.8         1
 6 Mariva    aeroponic      7      6   99.2         0
 7 Costanera aeroponic      1      5   24.1         0
 8 Costanera bed            3      5   78           1
 9 Costanera hydroponic     4      5   42           1
10 Mariva    aeroponic      5      5   72.6         0
# i 14 more rows
\end{verbatim}

und hier für zwei Kriterien (wobei letzteres Kriterium nur bei gleichen
Werten in erstem Kriterium sortiert wird)

\begin{Shaded}
\begin{Highlighting}[]
\NormalTok{pot.sub }\SpecialCharTok{\%\textgreater{}\%} \FunctionTok{arrange}\NormalTok{(}\FunctionTok{desc}\NormalTok{(tubers),}\FunctionTok{desc}\NormalTok{(weight))}
\end{Highlighting}
\end{Shaded}

\begin{verbatim}
# A tibble: 24 x 6
# Groups:   variety, method [12]
   variety   method     plant tubers weight infection
   <chr>     <chr>      <dbl>  <dbl>  <dbl>     <dbl>
 1 Costanera pot            6      7  139.          0
 2 Costanera bed            4      7  137.          1
 3 Mariva    pot            5      7  104.          0
 4 Costanera aeroponic      5      7   42           0
 5 Mariva    aeroponic      7      6   99.2         0
 6 Costanera hydroponic     8      6   31.8         1
 7 Unica     pot            9      5   97.7         1
 8 Costanera bed            3      5   78           1
 9 Mariva    bed            8      5   77.3         0
10 Mariva    aeroponic      5      5   72.6         0
# i 14 more rows
\end{verbatim}

\hypertarget{slice}{%
\paragraph{slice}\label{slice}}

Mit der Funktion \texttt{slice} selektieren wir die Zeilen. Hier Zeile 5
bis 10 im Datensatz \texttt{pot}.

\begin{Shaded}
\begin{Highlighting}[]
\NormalTok{pot }\SpecialCharTok{\%\textgreater{}\%} \FunctionTok{slice}\NormalTok{(}\DecValTok{5}\SpecialCharTok{:}\DecValTok{10}\NormalTok{)}
\end{Highlighting}
\end{Shaded}

\begin{verbatim}
  variety method plant tubers weight infection
1   Unica    pot     5      2   11.3         1
2   Unica    pot     6      2   17.8         0
3   Unica    pot     7      3   28.1         0
4   Unica    pot     8      6   33.0         0
5   Unica    pot     9      9   81.5         0
6   Unica    pot    10      6   71.1         0
\end{verbatim}

Die Funktion \texttt{slice\_head(n=3)} zeigt die ersten drei Zeilen des
Datensatzes an.

\begin{Shaded}
\begin{Highlighting}[]
\NormalTok{pot }\SpecialCharTok{\%\textgreater{}\%} \FunctionTok{slice\_head}\NormalTok{(}\AttributeTok{n=}\DecValTok{3}\NormalTok{)}
\end{Highlighting}
\end{Shaded}

\begin{verbatim}
  variety method plant tubers weight infection
1   Unica    pot     1      9  209.9         0
2   Unica    pot     2      3  248.4         0
3   Unica    pot     3      4   53.6         0
\end{verbatim}

Die Funktion \texttt{slice\_tail(n=3)} zeigt die letzten drei Zeilen des
Datensatzes an.

\begin{Shaded}
\begin{Highlighting}[]
\NormalTok{pot }\SpecialCharTok{\%\textgreater{}\%} \FunctionTok{slice\_tail}\NormalTok{(}\AttributeTok{n=}\DecValTok{3}\NormalTok{)}
\end{Highlighting}
\end{Shaded}

\begin{verbatim}
    variety    method plant tubers weight infection
1 Costanera aeroponic     8      6   29.1         1
2 Costanera aeroponic     9      4   17.6         0
3 Costanera aeroponic    10      7   36.9         0
\end{verbatim}

Alternativ könnte man auch die Funtionen \texttt{head(pot)} und
\texttt{tail(pot)} nutzen.

Die Funktion \texttt{slice\_sample(n=3)} wählt zufällig drei
Beobachtungen aus.

\begin{Shaded}
\begin{Highlighting}[]
\NormalTok{pot }\SpecialCharTok{\%\textgreater{}\%} \FunctionTok{slice\_sample}\NormalTok{(}\AttributeTok{n=}\DecValTok{3}\NormalTok{)}
\end{Highlighting}
\end{Shaded}

\begin{verbatim}
    variety     method plant tubers weight infection
1     Unica        bed     3      2   18.2         1
2 Costanera        bed     2      4  118.5         1
3     Unica hydroponic     3      2   58.7         0
\end{verbatim}

Wenn man immer das gleiche Set an Proben zufällig auswählen möchte, d.h.
ein reproduzierbares Ergebnis erzielen möchte, dann setzt man vorher den
\texttt{seed}. Die Zahl (hier 123) kann beliebig gewählt werden.

\begin{Shaded}
\begin{Highlighting}[]
\FunctionTok{set.seed}\NormalTok{(}\DecValTok{123}\NormalTok{)}
\NormalTok{pot }\SpecialCharTok{\%\textgreater{}\%} \FunctionTok{slice\_sample}\NormalTok{(}\AttributeTok{n=}\DecValTok{3}\NormalTok{)}
\end{Highlighting}
\end{Shaded}

\begin{verbatim}
    variety     method plant tubers weight infection
1    Mariva        bed     5      3   37.8         1
2 Costanera hydroponic     3      7   18.6         1
3    Mariva        bed     9      6   99.4         0
\end{verbatim}

Die Funktion \texttt{slice\_max(n=3)} wählt die drei Beobachtungen mit
dem höchsten Gewicht aus.

\begin{Shaded}
\begin{Highlighting}[]
\NormalTok{pot }\SpecialCharTok{\%\textgreater{}\%} \FunctionTok{slice\_max}\NormalTok{(weight, }\AttributeTok{n=}\DecValTok{3}\NormalTok{)}
\end{Highlighting}
\end{Shaded}

\begin{verbatim}
  variety    method plant tubers weight infection
1  Mariva aeroponic     6     13  323.3         0
2   Unica aeroponic     9      5  265.4         0
3   Unica       pot     2      3  248.4         0
\end{verbatim}

\hypertarget{spalten}{%
\subsubsection{Spalten}\label{spalten}}

Häufig arbeitet man mit großen Datensätzen und vielen Variablen.

\hypertarget{select}{%
\paragraph{select}\label{select}}

Mit der Funktion \texttt{select} kann man Variablen auswählen, indem ich
sie durch Komme getrennt aufliste:

\begin{Shaded}
\begin{Highlighting}[]
\NormalTok{pot.sub }\SpecialCharTok{\%\textgreater{}\%} \FunctionTok{select}\NormalTok{(variety, method, tubers, weight)}
\end{Highlighting}
\end{Shaded}

\begin{verbatim}
# A tibble: 24 x 4
# Groups:   variety, method [12]
   variety   method     tubers weight
   <chr>     <chr>       <dbl>  <dbl>
 1 Costanera aeroponic       5   24.1
 2 Costanera aeroponic       7   42  
 3 Costanera bed             7  137. 
 4 Costanera bed             5   78  
 5 Costanera hydroponic      6   31.8
 6 Costanera hydroponic      5   42  
 7 Costanera pot             3  110. 
 8 Costanera pot             7  139. 
 9 Mariva    aeroponic       6   99.2
10 Mariva    aeroponic       5   72.6
# i 14 more rows
\end{verbatim}

oder von \texttt{variety} bis \texttt{weight}

\begin{Shaded}
\begin{Highlighting}[]
\NormalTok{pot.sub }\SpecialCharTok{\%\textgreater{}\%} \FunctionTok{select}\NormalTok{(variety}\SpecialCharTok{:}\NormalTok{weight)}
\end{Highlighting}
\end{Shaded}

\begin{verbatim}
# A tibble: 24 x 5
# Groups:   variety, method [12]
   variety   method     plant tubers weight
   <chr>     <chr>      <dbl>  <dbl>  <dbl>
 1 Costanera aeroponic      1      5   24.1
 2 Costanera aeroponic      5      7   42  
 3 Costanera bed            4      7  137. 
 4 Costanera bed            3      5   78  
 5 Costanera hydroponic     8      6   31.8
 6 Costanera hydroponic     4      5   42  
 7 Costanera pot            5      3  110. 
 8 Costanera pot            6      7  139. 
 9 Mariva    aeroponic      7      6   99.2
10 Mariva    aeroponic      5      5   72.6
# i 14 more rows
\end{verbatim}

wie oben nur ohne \texttt{plant}

\begin{Shaded}
\begin{Highlighting}[]
\NormalTok{pot.sub }\SpecialCharTok{\%\textgreater{}\%} 
  \FunctionTok{select}\NormalTok{(variety}\SpecialCharTok{:}\NormalTok{weight)}\SpecialCharTok{\%\textgreater{}\%} 
  \FunctionTok{select}\NormalTok{(}\SpecialCharTok{!}\NormalTok{plant)}
\end{Highlighting}
\end{Shaded}

\begin{verbatim}
# A tibble: 24 x 4
# Groups:   variety, method [12]
   variety   method     tubers weight
   <chr>     <chr>       <dbl>  <dbl>
 1 Costanera aeroponic       5   24.1
 2 Costanera aeroponic       7   42  
 3 Costanera bed             7  137. 
 4 Costanera bed             5   78  
 5 Costanera hydroponic      6   31.8
 6 Costanera hydroponic      5   42  
 7 Costanera pot             3  110. 
 8 Costanera pot             7  139. 
 9 Mariva    aeroponic       6   99.2
10 Mariva    aeroponic       5   72.6
# i 14 more rows
\end{verbatim}

\begin{Shaded}
\begin{Highlighting}[]
\NormalTok{pot }\SpecialCharTok{\%\textgreater{}\%} 
  \FunctionTok{select}\NormalTok{(}\FunctionTok{ends\_with}\NormalTok{(}\StringTok{".f"}\NormalTok{)) }\SpecialCharTok{\%\textgreater{}\%} 
  \FunctionTok{slice\_sample}\NormalTok{(}\AttributeTok{n=}\DecValTok{8}\NormalTok{)}
\end{Highlighting}
\end{Shaded}

\begin{verbatim}
Dataframe mit 0 Spalten und 8 Zeilen
\end{verbatim}

\hypertarget{rename}{%
\subsubsection{rename}\label{rename}}

Variablen können umbenannt werden durch die Funktion \texttt{rename()}.

\begin{Shaded}
\begin{Highlighting}[]
\NormalTok{pot.sub }\SpecialCharTok{\%\textgreater{}\%} \FunctionTok{rename}\NormalTok{(}\AttributeTok{plant\_ID=}\NormalTok{plant)}
\end{Highlighting}
\end{Shaded}

\begin{verbatim}
# A tibble: 24 x 6
# Groups:   variety, method [12]
   variety   method     plant_ID tubers weight infection
   <chr>     <chr>         <dbl>  <dbl>  <dbl>     <dbl>
 1 Costanera aeroponic         1      5   24.1         0
 2 Costanera aeroponic         5      7   42           0
 3 Costanera bed               4      7  137.          1
 4 Costanera bed               3      5   78           1
 5 Costanera hydroponic        8      6   31.8         1
 6 Costanera hydroponic        4      5   42           1
 7 Costanera pot               5      3  110.          0
 8 Costanera pot               6      7  139.          0
 9 Mariva    aeroponic         7      6   99.2         0
10 Mariva    aeroponic         5      5   72.6         0
# i 14 more rows
\end{verbatim}

\begin{Shaded}
\begin{Highlighting}[]
\NormalTok{pot.sub}
\end{Highlighting}
\end{Shaded}

\begin{verbatim}
# A tibble: 24 x 6
# Groups:   variety, method [12]
   variety   method     plant tubers weight infection
   <chr>     <chr>      <dbl>  <dbl>  <dbl>     <dbl>
 1 Costanera aeroponic      1      5   24.1         0
 2 Costanera aeroponic      5      7   42           0
 3 Costanera bed            4      7  137.          1
 4 Costanera bed            3      5   78           1
 5 Costanera hydroponic     8      6   31.8         1
 6 Costanera hydroponic     4      5   42           1
 7 Costanera pot            5      3  110.          0
 8 Costanera pot            6      7  139.          0
 9 Mariva    aeroponic      7      6   99.2         0
10 Mariva    aeroponic      5      5   72.6         0
# i 14 more rows
\end{verbatim}

Auch hire müssen wir die Daten in einem neuen
\texttt{data.frame\ pot.sub1} speichern oder überschreiben
(\texttt{pot.sub=pot.sub\ \%\textgreater{}\%\ rename(plant\_ID=plant)}),
um auf die neue Variable zugreifen zu können.

\begin{Shaded}
\begin{Highlighting}[]
\NormalTok{pot.sub1}\OtherTok{=}\NormalTok{pot.sub }\SpecialCharTok{\%\textgreater{}\%} \FunctionTok{rename}\NormalTok{(}\AttributeTok{plant\_ID=}\NormalTok{plant)}
\NormalTok{pot.sub1}
\end{Highlighting}
\end{Shaded}

\begin{verbatim}
# A tibble: 24 x 6
# Groups:   variety, method [12]
   variety   method     plant_ID tubers weight infection
   <chr>     <chr>         <dbl>  <dbl>  <dbl>     <dbl>
 1 Costanera aeroponic         1      5   24.1         0
 2 Costanera aeroponic         5      7   42           0
 3 Costanera bed               4      7  137.          1
 4 Costanera bed               3      5   78           1
 5 Costanera hydroponic        8      6   31.8         1
 6 Costanera hydroponic        4      5   42           1
 7 Costanera pot               5      3  110.          0
 8 Costanera pot               6      7  139.          0
 9 Mariva    aeroponic         7      6   99.2         0
10 Mariva    aeroponic         5      5   72.6         0
# i 14 more rows
\end{verbatim}

\hypertarget{mutate}{%
\paragraph{mutate}\label{mutate}}

Eine neue Variable kann durch die Funktion \texttt{mutate} berechnet und
eingefügt werden.

Beispielsweise könnten wir das Gewicht in kg statt g angeben. Es wird
eine neue Variable erzeugt \texttt{weight\_kg} indem \texttt{weight}
durch 1000 geteilt wird.

\begin{Shaded}
\begin{Highlighting}[]
\NormalTok{pot.sub }\SpecialCharTok{\%\textgreater{}\%} 
  \FunctionTok{mutate}\NormalTok{(}\AttributeTok{weight\_kg=}\NormalTok{weight}\SpecialCharTok{/}\DecValTok{1000}\NormalTok{)}
\end{Highlighting}
\end{Shaded}

\begin{verbatim}
# A tibble: 24 x 7
# Groups:   variety, method [12]
   variety   method     plant tubers weight infection weight_kg
   <chr>     <chr>      <dbl>  <dbl>  <dbl>     <dbl>     <dbl>
 1 Costanera aeroponic      1      5   24.1         0    0.0241
 2 Costanera aeroponic      5      7   42           0    0.042 
 3 Costanera bed            4      7  137.          1    0.137 
 4 Costanera bed            3      5   78           1    0.078 
 5 Costanera hydroponic     8      6   31.8         1    0.0318
 6 Costanera hydroponic     4      5   42           1    0.042 
 7 Costanera pot            5      3  110.          0    0.110 
 8 Costanera pot            6      7  139.          0    0.139 
 9 Mariva    aeroponic      7      6   99.2         0    0.0992
10 Mariva    aeroponic      5      5   72.6         0    0.0726
# i 14 more rows
\end{verbatim}

Damit diese neue Variable im Datensatz nicht nur erscheint, sondern auch
zugreifbar ist, speichere ich den Datensatz unter einem neuen Namen ab.
Man könnte ihn auch überschreiben. Ausserdem noch zwei Beispiele für
eine log- und Wurzel-Transformation

\begin{Shaded}
\begin{Highlighting}[]
\NormalTok{pot.sub1}\OtherTok{=}\NormalTok{pot.sub }\SpecialCharTok{\%\textgreater{}\%} 
  \FunctionTok{mutate}\NormalTok{(}\AttributeTok{weight\_kg=}\NormalTok{weight}\SpecialCharTok{/}\DecValTok{1000}\NormalTok{,}
         \AttributeTok{weight.sqrt=}\FunctionTok{sqrt}\NormalTok{(weight),}
         \AttributeTok{weight.l=}\FunctionTok{log}\NormalTok{(weight),}
         \AttributeTok{tubers.l=}\FunctionTok{log}\NormalTok{(tubers))}
\NormalTok{pot.sub1}
\end{Highlighting}
\end{Shaded}

\begin{verbatim}
# A tibble: 24 x 10
# Groups:   variety, method [12]
   variety   method plant tubers weight infection weight_kg weight.sqrt weight.l
   <chr>     <chr>  <dbl>  <dbl>  <dbl>     <dbl>     <dbl>       <dbl>    <dbl>
 1 Costanera aerop~     1      5   24.1         0    0.0241        4.91     3.18
 2 Costanera aerop~     5      7   42           0    0.042         6.48     3.74
 3 Costanera bed        4      7  137.          1    0.137        11.7      4.92
 4 Costanera bed        3      5   78           1    0.078         8.83     4.36
 5 Costanera hydro~     8      6   31.8         1    0.0318        5.64     3.46
 6 Costanera hydro~     4      5   42           1    0.042         6.48     3.74
 7 Costanera pot        5      3  110.          0    0.110        10.5      4.70
 8 Costanera pot        6      7  139.          0    0.139        11.8      4.93
 9 Mariva    aerop~     7      6   99.2         0    0.0992        9.96     4.60
10 Mariva    aerop~     5      5   72.6         0    0.0726        8.52     4.28
# i 14 more rows
# i 1 more variable: tubers.l <dbl>
\end{verbatim}

\hypertarget{sec-mutate}{%
\subparagraph{Character Variable in Faktor umwandeln}\label{sec-mutate}}

Beim Importieren von Daten werden Variablen häufig als
\texttt{character} eingelesen (oder als \texttt{numeric}), die für die
spätere Analyse aber als Faktor benötigt werden. Wir können diese
Variablen in Faktoren umwandeln, können diese überschreiben oder (wie
unten gezeigt) geben ihnen einen neuen Namen mit dem Appendix ``.f''.

\begin{Shaded}
\begin{Highlighting}[]
\NormalTok{pot}\OtherTok{=}\NormalTok{pot }\SpecialCharTok{\%\textgreater{}\%} 
    \FunctionTok{mutate}\NormalTok{(}\AttributeTok{variety.f=}\FunctionTok{as.factor}\NormalTok{(variety),}
           \AttributeTok{method.f=}\FunctionTok{as.factor}\NormalTok{(method),}
           \AttributeTok{infection.f=}\FunctionTok{as.factor}\NormalTok{(infection))}
\end{Highlighting}
\end{Shaded}

alternativer Code zu oben:

\begin{Shaded}
\begin{Highlighting}[]
\NormalTok{pot}\SpecialCharTok{$}\NormalTok{variety.f}\OtherTok{=}\FunctionTok{as.factor}\NormalTok{(pot}\SpecialCharTok{$}\NormalTok{variety)}
\NormalTok{pot}\SpecialCharTok{$}\NormalTok{method.f}\OtherTok{=}\FunctionTok{as.factor}\NormalTok{(pot}\SpecialCharTok{$}\NormalTok{method)}
\NormalTok{pot}\SpecialCharTok{$}\NormalTok{infection.f}\OtherTok{=}\FunctionTok{as.factor}\NormalTok{(pot}\SpecialCharTok{$}\NormalTok{infection)}
\FunctionTok{str}\NormalTok{(pot)}
\end{Highlighting}
\end{Shaded}

Mit \texttt{across(where(is.character),as.factor)} werden alle
Variablen, die als Charakter eingelesen sind, in einen Faktor
umgewandelt.

\begin{Shaded}
\begin{Highlighting}[]
\NormalTok{pot2}\OtherTok{\textless{}{-}}\FunctionTok{read.xlsx}\NormalTok{(}\StringTok{"potato.xlsx"}\NormalTok{, }\AttributeTok{sheet=}\DecValTok{1}\NormalTok{)}
\FunctionTok{str}\NormalTok{(pot2)}
\end{Highlighting}
\end{Shaded}

\begin{verbatim}
'data.frame':   478 obs. of  6 variables:
 $ variety  : chr  "Unica" "Unica" "Unica" "Unica" ...
 $ method   : chr  "pot" "pot" "pot" "pot" ...
 $ plant    : num  1 2 3 4 5 6 7 8 9 10 ...
 $ tubers   : num  9 3 4 4 2 2 3 6 9 6 ...
 $ weight   : num  209.9 248.4 53.6 77.2 11.3 ...
 $ infection: num  0 0 0 0 1 0 0 0 0 0 ...
\end{verbatim}

\begin{Shaded}
\begin{Highlighting}[]
\NormalTok{pot3}\OtherTok{\textless{}{-}}\NormalTok{ pot2 }\SpecialCharTok{\%\textgreater{}\%} \FunctionTok{mutate}\NormalTok{(}\FunctionTok{across}\NormalTok{(}\FunctionTok{where}\NormalTok{(is.character),as.factor))}
\FunctionTok{str}\NormalTok{(pot3)}
\end{Highlighting}
\end{Shaded}

\begin{verbatim}
'data.frame':   478 obs. of  6 variables:
 $ variety  : Factor w/ 3 levels "Costanera","Mariva",..: 3 3 3 3 3 3 3 3 3 3 ...
 $ method   : Factor w/ 4 levels "aeroponic","bed",..: 4 4 4 4 4 4 4 4 4 4 ...
 $ plant    : num  1 2 3 4 5 6 7 8 9 10 ...
 $ tubers   : num  9 3 4 4 2 2 3 6 9 6 ...
 $ weight   : num  209.9 248.4 53.6 77.2 11.3 ...
 $ infection: num  0 0 0 0 1 0 0 0 0 0 ...
\end{verbatim}

Bitte beachten, dass \texttt{infection} damit nicht als Faktor
umgewandelt wird, da es als numerische Variable eingelesen wurde. Wir
müssen also immer nochmal die Struktur überprüfen und ggfls.
nachjustieren.

\begin{Shaded}
\begin{Highlighting}[]
\NormalTok{pot4}\OtherTok{\textless{}{-}}\NormalTok{ pot2 }\SpecialCharTok{\%\textgreater{}\%} \FunctionTok{mutate}\NormalTok{(}\FunctionTok{across}\NormalTok{(}\FunctionTok{where}\NormalTok{(is.character),as.factor),}
                       \AttributeTok{infection=}\FunctionTok{as.factor}\NormalTok{(infection))}
\FunctionTok{str}\NormalTok{(pot4)}
\end{Highlighting}
\end{Shaded}

\begin{verbatim}
'data.frame':   478 obs. of  6 variables:
 $ variety  : Factor w/ 3 levels "Costanera","Mariva",..: 3 3 3 3 3 3 3 3 3 3 ...
 $ method   : Factor w/ 4 levels "aeroponic","bed",..: 4 4 4 4 4 4 4 4 4 4 ...
 $ plant    : num  1 2 3 4 5 6 7 8 9 10 ...
 $ tubers   : num  9 3 4 4 2 2 3 6 9 6 ...
 $ weight   : num  209.9 248.4 53.6 77.2 11.3 ...
 $ infection: Factor w/ 2 levels "0","1": 1 1 1 1 2 1 1 1 1 1 ...
\end{verbatim}

Der große Vorteil von \texttt{dplyr} ist, dass ihr alle Schritte in
einen Code schreiben und damit ihn gut nachvollziehen könnt.

\begin{Shaded}
\begin{Highlighting}[]
\NormalTok{pot1}\OtherTok{=}\NormalTok{pot}\SpecialCharTok{\%\textgreater{}\%} 
  \FunctionTok{filter}\NormalTok{(variety}\SpecialCharTok{==}\FunctionTok{c}\NormalTok{(}\StringTok{"Mariva"}\NormalTok{, }\StringTok{"Costanera"}\NormalTok{), tubers}\SpecialCharTok{\textgreater{}}\DecValTok{7}\NormalTok{) }\SpecialCharTok{\%\textgreater{}\%} 
  \FunctionTok{mutate}\NormalTok{(}\AttributeTok{weight\_kg=}\NormalTok{weight}\SpecialCharTok{/}\DecValTok{1000}\NormalTok{, }
         \AttributeTok{variety.f=}\FunctionTok{as.factor}\NormalTok{(variety),}
         \AttributeTok{method.f=}\FunctionTok{as.factor}\NormalTok{(method),}
         \AttributeTok{infection.f=}\FunctionTok{as.factor}\NormalTok{(infection)) }\SpecialCharTok{\%\textgreater{}\%} 
  \FunctionTok{select}\NormalTok{(}\SpecialCharTok{!}\FunctionTok{c}\NormalTok{(plant,infection, weight))}
\NormalTok{pot1}
\end{Highlighting}
\end{Shaded}

\begin{verbatim}
     variety     method tubers variety.f   method.f infection.f weight_kg
1     Mariva        bed      9    Mariva        bed           0    0.2274
2     Mariva        bed     10    Mariva        bed           0    0.0338
3     Mariva        bed      8    Mariva        bed           0    0.0948
4  Costanera        pot      8 Costanera        pot           1    0.1624
5  Costanera hydroponic      8 Costanera hydroponic           1    0.0486
6  Costanera  aeroponic      9 Costanera  aeroponic           0    0.0781
7     Mariva        pot     10    Mariva        pot           0    0.0773
8     Mariva  aeroponic      8    Mariva  aeroponic           1    0.1205
9  Costanera        pot      8 Costanera        pot           0    0.1449
10 Costanera        pot      9 Costanera        pot           0    0.1391
11    Mariva        bed     11    Mariva        bed           0    0.1182
12    Mariva        bed      8    Mariva        bed           1    0.1250
13    Mariva  aeroponic     10    Mariva  aeroponic           0    0.1883
14    Mariva  aeroponic      8    Mariva  aeroponic           0    0.1656
15    Mariva        pot     10    Mariva        pot           0    0.0644
16    Mariva        bed     10    Mariva        bed           1    0.1069
17    Mariva        bed      8    Mariva        bed           1    0.1332
18    Mariva        bed     10    Mariva        bed           0    0.1183
19    Mariva  aeroponic      9    Mariva  aeroponic           0    0.0759
20    Mariva  aeroponic      8    Mariva  aeroponic           0    0.1684
21 Costanera        pot      8 Costanera        pot           0    0.1531
22 Costanera        bed      8 Costanera        bed           1    0.0494
23 Costanera hydroponic      8 Costanera hydroponic           1    0.0253
\end{verbatim}

\hypertarget{deskriptive-statistik}{%
\subsection{Deskriptive Statistik}\label{deskriptive-statistik}}

\begin{itemize}
\tightlist
\item
  Wir unterscheiden zwischen qualitativen (kategorialen) und
  quantitativen (numerischen) Daten.
\item
  Qualitative Daten können weiter differenziert werden in

  \begin{itemize}
  \tightlist
  \item
    nominale Daten (ohne Rangordnung), z.B. Geschlecht (m, w),
    Blutgruppe, Sorte und Augenfarbe
  \item
    ordinale Daten (mit Rangordnung), z.B. Eignung (gut, mittel,
    schlecht), Platzierung bei einem Rennen, Boniturnoten
  \end{itemize}
\item
  Quantitative Daten können weiter unterschieden werden in

  \begin{itemize}
  \tightlist
  \item
    diskrete (ganzzahlig) Daten, z.B. Anzahl Blattläuse, Anzahl
    Nachkommen
  \item
    stetige Daten, z.B. Ertrag, pH-Wert, Körpergröße, Parasitierungsrate
  \end{itemize}
\end{itemize}

\hypertarget{qualitative-daten}{%
\subsection{Qualitative Daten}\label{qualitative-daten}}

Qualitative Variablen sind in unserem Beispiel die Sorte, der
Krankheitsbefall und die Methode. Diese Daten beschreiben wir durch
\emph{Häufigkeitstabellen (Kontingenztabellen)}, die angeben, wie häufig
eine Merkmalsausprägung bzw. -kombination in unserem Datensatz vorkommt.
Wir nutzen die Funktion \texttt{count()}aus dem Package \texttt{dplyr}

\begin{Shaded}
\begin{Highlighting}[]
\NormalTok{pot }\SpecialCharTok{\%\textgreater{}\%} \FunctionTok{count}\NormalTok{(variety)}
\end{Highlighting}
\end{Shaded}

\begin{verbatim}
    variety   n
1 Costanera 158
2    Mariva 160
3     Unica 160
\end{verbatim}

oder die Funktion \texttt{table()}.

\begin{Shaded}
\begin{Highlighting}[]
\FunctionTok{table}\NormalTok{(pot}\SpecialCharTok{$}\NormalTok{variety) }
\end{Highlighting}
\end{Shaded}

\begin{verbatim}

Costanera    Mariva     Unica 
      158       160       160 
\end{verbatim}

Die Funktion \texttt{prop.table()} berechnet uns die relativen Anteile
jeder Merkmalsausprägung bzw. -kombination.

\begin{Shaded}
\begin{Highlighting}[]
\NormalTok{pot }\SpecialCharTok{\%\textgreater{}\%} \FunctionTok{count}\NormalTok{(variety) }\SpecialCharTok{\%\textgreater{}\%} 
  \FunctionTok{mutate}\NormalTok{(}\AttributeTok{prop =} \FunctionTok{prop.table}\NormalTok{(n))}
\end{Highlighting}
\end{Shaded}

\begin{verbatim}
    variety   n      prop
1 Costanera 158 0.3305439
2    Mariva 160 0.3347280
3     Unica 160 0.3347280
\end{verbatim}

\begin{Shaded}
\begin{Highlighting}[]
\FunctionTok{prop.table}\NormalTok{(}\FunctionTok{table}\NormalTok{(pot}\SpecialCharTok{$}\NormalTok{variety)) }\CommentTok{\# relativ, i.e. Anteil der Beobachtungen an der Gesamtzahl der Beobachtungen }
\end{Highlighting}
\end{Shaded}

\begin{verbatim}

Costanera    Mariva     Unica 
0.3305439 0.3347280 0.3347280 
\end{verbatim}

Häufigkeitstabellen können für 2 Kombinationen (Merkmale) erstellt
werden, indem man beide Variablen in der Funktion \texttt{count()} oder
\texttt{table()} angibt.

\begin{Shaded}
\begin{Highlighting}[]
\NormalTok{pot }\SpecialCharTok{\%\textgreater{}\%} \FunctionTok{count}\NormalTok{(variety, infection) }\SpecialCharTok{\%\textgreater{}\%} 
  \FunctionTok{mutate}\NormalTok{(}\AttributeTok{prop =} \FunctionTok{prop.table}\NormalTok{(n))}
\end{Highlighting}
\end{Shaded}

\begin{verbatim}
    variety infection   n       prop
1 Costanera         0  77 0.16108787
2 Costanera         1  81 0.16945607
3    Mariva         0 119 0.24895397
4    Mariva         1  41 0.08577406
5     Unica         0 122 0.25523013
6     Unica         1  38 0.07949791
\end{verbatim}

\begin{Shaded}
\begin{Highlighting}[]
\FunctionTok{table}\NormalTok{(pot}\SpecialCharTok{$}\NormalTok{variety, pot}\SpecialCharTok{$}\NormalTok{infection)}
\end{Highlighting}
\end{Shaded}

\begin{verbatim}
           
              0   1
  Costanera  77  81
  Mariva    119  41
  Unica     122  38
\end{verbatim}

\begin{Shaded}
\begin{Highlighting}[]
\FunctionTok{prop.table}\NormalTok{(}\FunctionTok{table}\NormalTok{(pot}\SpecialCharTok{$}\NormalTok{variety, pot}\SpecialCharTok{$}\NormalTok{infection)) }\CommentTok{\# relative Häufigkeit}
\end{Highlighting}
\end{Shaded}

\begin{verbatim}
           
                     0          1
  Costanera 0.16108787 0.16945607
  Mariva    0.24895397 0.08577406
  Unica     0.25523013 0.07949791
\end{verbatim}

Häufigkeitstabelle für 3 Kombinationen

\begin{Shaded}
\begin{Highlighting}[]
\NormalTok{pot }\SpecialCharTok{\%\textgreater{}\%} \FunctionTok{count}\NormalTok{(variety, method, infection) }\SpecialCharTok{\%\textgreater{}\%} 
  \FunctionTok{mutate}\NormalTok{(}\AttributeTok{prop =} \FunctionTok{prop.table}\NormalTok{(n))}
\end{Highlighting}
\end{Shaded}

\begin{verbatim}
     variety     method infection  n       prop
1  Costanera  aeroponic         0 28 0.05857741
2  Costanera  aeroponic         1 12 0.02510460
3  Costanera        bed         0  9 0.01882845
4  Costanera        bed         1 31 0.06485356
5  Costanera hydroponic         0 11 0.02301255
6  Costanera hydroponic         1 27 0.05648536
7  Costanera        pot         0 29 0.06066946
8  Costanera        pot         1 11 0.02301255
9     Mariva  aeroponic         0 27 0.05648536
10    Mariva  aeroponic         1 13 0.02719665
11    Mariva        bed         0 29 0.06066946
12    Mariva        bed         1 11 0.02301255
13    Mariva hydroponic         0 31 0.06485356
14    Mariva hydroponic         1  9 0.01882845
15    Mariva        pot         0 32 0.06694561
16    Mariva        pot         1  8 0.01673640
17     Unica  aeroponic         0 30 0.06276151
18     Unica  aeroponic         1 10 0.02092050
19     Unica        bed         0 31 0.06485356
20     Unica        bed         1  9 0.01882845
21     Unica hydroponic         0 32 0.06694561
22     Unica hydroponic         1  8 0.01673640
23     Unica        pot         0 29 0.06066946
24     Unica        pot         1 11 0.02301255
\end{verbatim}

\begin{Shaded}
\begin{Highlighting}[]
\FunctionTok{table}\NormalTok{(pot}\SpecialCharTok{$}\NormalTok{variety, pot}\SpecialCharTok{$}\NormalTok{method, pot}\SpecialCharTok{$}\NormalTok{infection)}
\end{Highlighting}
\end{Shaded}

\begin{verbatim}
, ,  = 0

           
            aeroponic bed hydroponic pot
  Costanera        28   9         11  29
  Mariva           27  29         31  32
  Unica            30  31         32  29

, ,  = 1

           
            aeroponic bed hydroponic pot
  Costanera        12  31         27  11
  Mariva           13  11          9   8
  Unica            10   9          8  11
\end{verbatim}

\begin{Shaded}
\begin{Highlighting}[]
\FunctionTok{ftable}\NormalTok{(pot}\SpecialCharTok{$}\NormalTok{variety, pot}\SpecialCharTok{$}\NormalTok{method, pot}\SpecialCharTok{$}\NormalTok{infection)}
\end{Highlighting}
\end{Shaded}

\begin{verbatim}
                       0  1
                           
Costanera aeroponic   28 12
          bed          9 31
          hydroponic  11 27
          pot         29 11
Mariva    aeroponic   27 13
          bed         29 11
          hydroponic  31  9
          pot         32  8
Unica     aeroponic   30 10
          bed         31  9
          hydroponic  32  8
          pot         29 11
\end{verbatim}

\hypertarget{quantitative-daten}{%
\subsection{Quantitative Daten}\label{quantitative-daten}}

\begin{itemize}
\tightlist
\item
  Arithmetischer Mittelwert \texttt{mean()}
\item
  Median \texttt{median()}: Wert, der an der mittleren (zentralen)
  Stelle steht, wenn man die Werte der Größe nach sortiert
\item
  Median besser als arithmetischer Mittelwert bei:

  \begin{itemize}
  \tightlist
  \item
    ordinalskalierten Beobachtungen
  \item
    geringem Stichprobenumfang
  \item
    asymmetrischen Verteilungen
  \item
    Verdacht auf Ausreißer
  \end{itemize}
\end{itemize}

\begin{Shaded}
\begin{Highlighting}[]
\NormalTok{pot }\SpecialCharTok{\%\textgreater{}\%} 
  \FunctionTok{summarise}\NormalTok{(}\AttributeTok{tubers\_avg=}\FunctionTok{mean}\NormalTok{(tubers))}
\end{Highlighting}
\end{Shaded}

\begin{verbatim}
  tubers_avg
1   4.721757
\end{verbatim}

\begin{Shaded}
\begin{Highlighting}[]
\NormalTok{pot }\SpecialCharTok{\%\textgreater{}\%} 
  \FunctionTok{summarise}\NormalTok{(}\AttributeTok{tubers\_avg=}\FunctionTok{mean}\NormalTok{(tubers),}
            \AttributeTok{tubers\_med=}\FunctionTok{median}\NormalTok{(tubers))}
\end{Highlighting}
\end{Shaded}

\begin{verbatim}
  tubers_avg tubers_med
1   4.721757          4
\end{verbatim}

\begin{Shaded}
\begin{Highlighting}[]
\FunctionTok{mean}\NormalTok{(pot}\SpecialCharTok{$}\NormalTok{tubers)}
\end{Highlighting}
\end{Shaded}

\begin{verbatim}
[1] 4.721757
\end{verbatim}

\begin{Shaded}
\begin{Highlighting}[]
\FunctionTok{median}\NormalTok{(pot}\SpecialCharTok{$}\NormalTok{tubers)}
\end{Highlighting}
\end{Shaded}

\begin{verbatim}
[1] 4
\end{verbatim}

Maße für die Streuung der Daten:

\begin{itemize}
\tightlist
\item
  Varianz \texttt{var()}
\item
  Standardabweichung \texttt{sd()}
\item
  Standardabweichung in gleicher Einheit wie Mittelwert
\item
  Wenn Mittelwert und Standardabweichung einer normalverteilten
  Grundgesamtheit bekannt ist, kann die Wahrscheinlichkeit berechnet
  werden, mit der ein Wert auftritt.
\end{itemize}

\begin{Shaded}
\begin{Highlighting}[]
\NormalTok{pot }\SpecialCharTok{\%\textgreater{}\%} 
  \FunctionTok{summarise}\NormalTok{(}\AttributeTok{tubers\_avg=}\FunctionTok{mean}\NormalTok{(tubers),}
            \AttributeTok{tubers\_med=}\FunctionTok{median}\NormalTok{(tubers),}
            \AttributeTok{tubers\_var=}\FunctionTok{var}\NormalTok{(tubers),}
            \AttributeTok{tubers\_sd=}\FunctionTok{sd}\NormalTok{(tubers))}
\end{Highlighting}
\end{Shaded}

\begin{verbatim}
  tubers_avg tubers_med tubers_var tubers_sd
1   4.721757          4   5.190763  2.278324
\end{verbatim}

\begin{Shaded}
\begin{Highlighting}[]
\FunctionTok{var}\NormalTok{(pot}\SpecialCharTok{$}\NormalTok{tubers)}
\end{Highlighting}
\end{Shaded}

\begin{verbatim}
[1] 5.190763
\end{verbatim}

\begin{Shaded}
\begin{Highlighting}[]
\FunctionTok{sd}\NormalTok{(pot}\SpecialCharTok{$}\NormalTok{tubers)}
\end{Highlighting}
\end{Shaded}

\begin{verbatim}
[1] 2.278324
\end{verbatim}

Der Standardfehler des Mittelwertes (sem) beschreibt die Genauigkeit der
Berechnung des Stichproben-Mittelwertes.

\begin{itemize}
\tightlist
\item
  sem = sd/sqrt(n)
\item
  \texttt{std\ \textless{}-\ function(x)\ \{sd(x,\ na.rm=TRUE)/sqrt(length(na.omit(x)))\}}
\item
  kein Streuungsmaß der Stichprobe
\item
  je mehr Datenpunkte, desto genauer die Schätzung des Mittelwertes
\item
  Mittelwert ± 1 sem beschreibt den Wertebereich, in dem wir mit
  68\%iger Wahrscheinlichkeit den wahren Mittelwert erwarten
\item
  Mittelwert ± 1,96 sem 95\% Wahrscheinlichkeit i.e.~Konfidenzintervall
\item
  Mittelwert ± 2 sem 95,5\%
\item
  Mittelwert ± 3 sem 99,7\%
\end{itemize}

\begin{Shaded}
\begin{Highlighting}[]
\CommentTok{\# Funktion für den Standardfehler}
\NormalTok{std }\OtherTok{\textless{}{-}} \ControlFlowTok{function}\NormalTok{(x) \{}\FunctionTok{sd}\NormalTok{(x, }\AttributeTok{na.rm=}\ConstantTok{TRUE}\NormalTok{)}\SpecialCharTok{/}\FunctionTok{sqrt}\NormalTok{(}\FunctionTok{length}\NormalTok{(}\FunctionTok{na.omit}\NormalTok{(x)))\} }\CommentTok{\#muss nur einmal definiert werden}
\FunctionTok{std}\NormalTok{(pot}\SpecialCharTok{$}\NormalTok{tubers)}
\end{Highlighting}
\end{Shaded}

\begin{verbatim}
[1] 0.1042081
\end{verbatim}

weitere Maße zur beschreibenden Statistik:

\begin{itemize}
\tightlist
\item
  Minimum \texttt{min()}
\item
  Maximum \texttt{max()}
\item
  Wertebereich \texttt{range()}
\item
  Quantile \texttt{quantile()}
\item
  Varianzkoeffizient = CV = sd/mean
\end{itemize}

\begin{Shaded}
\begin{Highlighting}[]
\FunctionTok{min}\NormalTok{(pot}\SpecialCharTok{$}\NormalTok{tubers)}
\end{Highlighting}
\end{Shaded}

\begin{verbatim}
[1] 0
\end{verbatim}

\begin{Shaded}
\begin{Highlighting}[]
\FunctionTok{max}\NormalTok{(pot}\SpecialCharTok{$}\NormalTok{tubers)}
\end{Highlighting}
\end{Shaded}

\begin{verbatim}
[1] 13
\end{verbatim}

\begin{Shaded}
\begin{Highlighting}[]
\FunctionTok{range}\NormalTok{(pot}\SpecialCharTok{$}\NormalTok{tubers)}
\end{Highlighting}
\end{Shaded}

\begin{verbatim}
[1]  0 13
\end{verbatim}

\begin{Shaded}
\begin{Highlighting}[]
\FunctionTok{quantile}\NormalTok{(pot}\SpecialCharTok{$}\NormalTok{tubers)}
\end{Highlighting}
\end{Shaded}

\begin{verbatim}
  0%  25%  50%  75% 100% 
   0    3    4    6   13 
\end{verbatim}

\begin{Shaded}
\begin{Highlighting}[]
\FunctionTok{quantile}\NormalTok{(pot}\SpecialCharTok{$}\NormalTok{tubers, }\AttributeTok{p=}\FunctionTok{c}\NormalTok{(}\FloatTok{0.01}\NormalTok{, }\FloatTok{0.05}\NormalTok{, }\FloatTok{0.1}\NormalTok{, }\FloatTok{0.25}\NormalTok{, }\FloatTok{0.5}\NormalTok{, }\FloatTok{0.75}\NormalTok{, }\FloatTok{0.90}\NormalTok{, }\FloatTok{0.95}\NormalTok{, }\FloatTok{0.99}\NormalTok{))}
\end{Highlighting}
\end{Shaded}

\begin{verbatim}
 1%  5% 10% 25% 50% 75% 90% 95% 99% 
  0   2   2   3   4   6   8   9  11 
\end{verbatim}

\begin{Shaded}
\begin{Highlighting}[]
\CommentTok{\#Funktion für Variationskoeffizienten}
\NormalTok{CV }\OtherTok{\textless{}{-}} \ControlFlowTok{function}\NormalTok{(x) \{}\FunctionTok{sd}\NormalTok{(x, }\AttributeTok{na.rm=}\ConstantTok{TRUE}\NormalTok{)}\SpecialCharTok{/}\FunctionTok{mean}\NormalTok{(x, }\AttributeTok{na.rm=}\ConstantTok{TRUE}\NormalTok{)\} }
\end{Highlighting}
\end{Shaded}

q25 = \textasciitilde quantile(., 0.25),

\begin{Shaded}
\begin{Highlighting}[]
\NormalTok{pot }\SpecialCharTok{\%\textgreater{}\%} 
  \FunctionTok{summarise}\NormalTok{(}\AttributeTok{tubers\_avg=}\FunctionTok{mean}\NormalTok{(tubers),}
            \AttributeTok{tubers\_med=}\FunctionTok{median}\NormalTok{(tubers),}
            \AttributeTok{tubers\_var=}\FunctionTok{var}\NormalTok{(tubers),}
            \AttributeTok{tubers\_sd=}\FunctionTok{sd}\NormalTok{(tubers),}
            \AttributeTok{tubers\_std=}\FunctionTok{std}\NormalTok{(tubers),}
            \AttributeTok{tubers\_min=}\FunctionTok{min}\NormalTok{(tubers),}
            \AttributeTok{tubers\_max=}\FunctionTok{max}\NormalTok{(tubers),}
            \AttributeTok{tubers\_q25=}\FunctionTok{quantile}\NormalTok{(tubers, }\FloatTok{0.25}\NormalTok{),}
            \AttributeTok{tubers\_q75=}\FunctionTok{quantile}\NormalTok{(tubers, }\FloatTok{0.75}\NormalTok{))}
\end{Highlighting}
\end{Shaded}

\begin{verbatim}
  tubers_avg tubers_med tubers_var tubers_sd tubers_std tubers_min tubers_max
1   4.721757          4   5.190763  2.278324  0.1042081          0         13
  tubers_q25 tubers_q75
1          3          6
\end{verbatim}

\hypertarget{anwendungsbeispiele}{%
\subsubsection{Anwendungsbeispiele}\label{anwendungsbeispiele}}

Eine erste einfache Beschreibung der Daten kann mit der Funktion
\texttt{summary()} erfolgen. Hier sieht man jetzt den Unterschied im
Output zwischen variety (als \emph{character}) und variety.f (als
\emph{factor}).

\begin{Shaded}
\begin{Highlighting}[]
\FunctionTok{summary}\NormalTok{(pot)}
\end{Highlighting}
\end{Shaded}

\begin{verbatim}
   variety             method              plant           tubers      
 Length:478         Length:478         Min.   : 1.00   Min.   : 0.000  
 Class :character   Class :character   1st Qu.: 3.00   1st Qu.: 3.000  
 Mode  :character   Mode  :character   Median : 5.00   Median : 4.000  
                                       Mean   : 5.49   Mean   : 4.722  
                                       3rd Qu.: 8.00   3rd Qu.: 6.000  
                                       Max.   :10.00   Max.   :13.000  
     weight         infection          variety.f         method.f   infection.f
 Min.   :  0.00   Min.   :0.0000   Costanera:158   aeroponic :120   0:318      
 1st Qu.: 26.07   1st Qu.:0.0000   Mariva   :160   bed       :120   1:160      
 Median : 61.00   Median :0.0000   Unica    :160   hydroponic:118              
 Mean   : 72.77   Mean   :0.3347                   pot       :120              
 3rd Qu.:109.25   3rd Qu.:1.0000                                               
 Max.   :323.30   Max.   :1.0000                                               
\end{verbatim}

Möchte man eine beschreibende Statistik für jede numerische Variable
berechnen, kann die Funktion
\texttt{summarise\_if(is.numeric,\ mean,\ na.rm\ =\ TRUE)} genutzt
werden. Im Beispiel berechnen wir den Mittelwert für alle Variablen im
\texttt{data.frame\ pot}.

\begin{Shaded}
\begin{Highlighting}[]
\NormalTok{pot }\SpecialCharTok{\%\textgreater{}\%}
  \FunctionTok{summarise\_if}\NormalTok{(is.numeric, mean, }\AttributeTok{na.rm =} \ConstantTok{TRUE}\NormalTok{)}
\end{Highlighting}
\end{Shaded}

\begin{verbatim}
    plant   tubers   weight infection
1 5.48954 4.721757 72.77259  0.334728
\end{verbatim}

Häufig möchte man die beschreibende Statistik für ein oder mehrere
Gruppierungslevel berechnen. Bspw. das mittleres Gewicht je Sorte. Wir
nutzen hierfür die Funktion \texttt{group\_by()}:

\begin{Shaded}
\begin{Highlighting}[]
\NormalTok{pot }\SpecialCharTok{\%\textgreater{}\%} \FunctionTok{group\_by}\NormalTok{(variety) }\SpecialCharTok{\%\textgreater{}\%} 
  \FunctionTok{summarise}\NormalTok{(}\AttributeTok{weight\_avg=}\FunctionTok{mean}\NormalTok{(weight, }\AttributeTok{na.rm =} \ConstantTok{TRUE}\NormalTok{))}
\end{Highlighting}
\end{Shaded}

\begin{verbatim}
# A tibble: 3 x 2
  variety   weight_avg
  <chr>          <dbl>
1 Costanera       74.1
2 Mariva          74.7
3 Unica           69.6
\end{verbatim}

Bsp.: mittleres Gewicht je Sorte und Methode

\begin{Shaded}
\begin{Highlighting}[]
\NormalTok{pot }\SpecialCharTok{\%\textgreater{}\%} \FunctionTok{group\_by}\NormalTok{(variety, method) }\SpecialCharTok{\%\textgreater{}\%} 
  \FunctionTok{summarise}\NormalTok{(}\AttributeTok{weight\_avg=}\FunctionTok{mean}\NormalTok{(weight, }\AttributeTok{na.rm =} \ConstantTok{TRUE}\NormalTok{))}
\end{Highlighting}
\end{Shaded}

\begin{verbatim}
`summarise()` has grouped output by 'variety'. You can override using the
`.groups` argument.
\end{verbatim}

\begin{verbatim}
# A tibble: 12 x 3
# Groups:   variety [3]
   variety   method     weight_avg
   <chr>     <chr>           <dbl>
 1 Costanera aeroponic        40.2
 2 Costanera bed              87.5
 3 Costanera hydroponic       25.1
 4 Costanera pot             141. 
 5 Mariva    aeroponic        95.4
 6 Mariva    bed              96.8
 7 Mariva    hydroponic       14.3
 8 Mariva    pot              92.2
 9 Unica     aeroponic        88.5
10 Unica     bed              69.1
11 Unica     hydroponic       26.3
12 Unica     pot              94.6
\end{verbatim}

Bsp.: Mittelwert und Standardabweichung von Gewicht je Sorte und Methode

\begin{Shaded}
\begin{Highlighting}[]
\NormalTok{pot }\SpecialCharTok{\%\textgreater{}\%} \FunctionTok{group\_by}\NormalTok{(variety, method) }\SpecialCharTok{\%\textgreater{}\%} 
  \FunctionTok{summarise}\NormalTok{(}\AttributeTok{weight\_avg=}\FunctionTok{mean}\NormalTok{(weight, }\AttributeTok{na.rm =} \ConstantTok{TRUE}\NormalTok{),}
            \AttributeTok{weight\_sd=}\FunctionTok{sd}\NormalTok{(weight, }\AttributeTok{na.rm =} \ConstantTok{TRUE}\NormalTok{))}
\end{Highlighting}
\end{Shaded}

\begin{verbatim}
`summarise()` has grouped output by 'variety'. You can override using the
`.groups` argument.
\end{verbatim}

\begin{verbatim}
# A tibble: 12 x 4
# Groups:   variety [3]
   variety   method     weight_avg weight_sd
   <chr>     <chr>           <dbl>     <dbl>
 1 Costanera aeroponic        40.2     16.6 
 2 Costanera bed              87.5     32.6 
 3 Costanera hydroponic       25.1      8.17
 4 Costanera pot             141.      22.2 
 5 Mariva    aeroponic        95.4     54.8 
 6 Mariva    bed              96.8     54.8 
 7 Mariva    hydroponic       14.3      9.34
 8 Mariva    pot              92.2     27.4 
 9 Unica     aeroponic        88.5     70.8 
10 Unica     bed              69.1     36.4 
11 Unica     hydroponic       26.3     14.1 
12 Unica     pot              94.6     60.3 
\end{verbatim}

Hier ein Code für eine Übersichtstabelle zur Beschreibung der Daten:

\begin{Shaded}
\begin{Highlighting}[]
\FunctionTok{library}\NormalTok{(tidyr)}
\NormalTok{pot }\SpecialCharTok{\%\textgreater{}\%} \FunctionTok{summarise}\NormalTok{(}\FunctionTok{across}\NormalTok{(}\FunctionTok{where}\NormalTok{(is.numeric), }\AttributeTok{.fns =} 
                     \FunctionTok{list}\NormalTok{(}\AttributeTok{min =}\NormalTok{ min,}
                          \AttributeTok{median =}\NormalTok{ median,}
                          \AttributeTok{mean =}\NormalTok{ mean,}
                          \AttributeTok{stdev =}\NormalTok{ sd,}
                          \AttributeTok{q25 =} \SpecialCharTok{\textasciitilde{}}\FunctionTok{quantile}\NormalTok{(., }\FloatTok{0.25}\NormalTok{),}
                          \AttributeTok{q75 =} \SpecialCharTok{\textasciitilde{}}\FunctionTok{quantile}\NormalTok{(., }\FloatTok{0.75}\NormalTok{),}
                          \AttributeTok{max =}\NormalTok{ max, }
                          \AttributeTok{n=}\NormalTok{length))) }\SpecialCharTok{\%\textgreater{}\%}
  \FunctionTok{pivot\_longer}\NormalTok{(}\FunctionTok{everything}\NormalTok{(), }\AttributeTok{names\_sep=}\StringTok{\textquotesingle{}\_\textquotesingle{}}\NormalTok{, }\AttributeTok{names\_to=}\FunctionTok{c}\NormalTok{(}\StringTok{\textquotesingle{}variable\textquotesingle{}}\NormalTok{, }\StringTok{\textquotesingle{}.value\textquotesingle{}}\NormalTok{))}
\end{Highlighting}
\end{Shaded}

\begin{verbatim}
# A tibble: 4 x 9
  variable    min median   mean  stdev   q25   q75   max     n
  <chr>     <dbl>  <dbl>  <dbl>  <dbl> <dbl> <dbl> <dbl> <int>
1 plant         1      5  5.49   2.87    3      8    10    478
2 tubers        0      4  4.72   2.28    3      6    13    478
3 weight        0     61 72.8   53.6    26.1  109.  323.   478
4 infection     0      0  0.335  0.472   0      1     1    478
\end{verbatim}

und hier nur für \texttt{tubers} und \texttt{weight}:

\begin{Shaded}
\begin{Highlighting}[]
\NormalTok{pot }\SpecialCharTok{\%\textgreater{}\%} \FunctionTok{summarise}\NormalTok{(}\FunctionTok{across}\NormalTok{(}\FunctionTok{c}\NormalTok{(}\StringTok{"tubers"}\NormalTok{, }\StringTok{"weight"}\NormalTok{), }\AttributeTok{.fns =} 
                     \FunctionTok{list}\NormalTok{(}\AttributeTok{min =}\NormalTok{ min,}
                          \AttributeTok{median =}\NormalTok{ median,}
                          \AttributeTok{mean =}\NormalTok{ mean,}
                          \AttributeTok{stdev =}\NormalTok{ sd,}
                          \AttributeTok{q25 =} \SpecialCharTok{\textasciitilde{}}\FunctionTok{quantile}\NormalTok{(., }\FloatTok{0.25}\NormalTok{),}
                          \AttributeTok{q75 =} \SpecialCharTok{\textasciitilde{}}\FunctionTok{quantile}\NormalTok{(., }\FloatTok{0.75}\NormalTok{),}
                          \AttributeTok{max =}\NormalTok{ max, }
                          \AttributeTok{n=}\NormalTok{length))) }\SpecialCharTok{\%\textgreater{}\%}
  \FunctionTok{pivot\_longer}\NormalTok{(}\FunctionTok{everything}\NormalTok{(), }\AttributeTok{names\_sep=}\StringTok{\textquotesingle{}\_\textquotesingle{}}\NormalTok{, }\AttributeTok{names\_to=}\FunctionTok{c}\NormalTok{(}\StringTok{\textquotesingle{}variable\textquotesingle{}}\NormalTok{, }\StringTok{\textquotesingle{}.value\textquotesingle{}}\NormalTok{))}
\end{Highlighting}
\end{Shaded}

\begin{verbatim}
# A tibble: 2 x 9
  variable   min median  mean stdev   q25   q75   max     n
  <chr>    <dbl>  <dbl> <dbl> <dbl> <dbl> <dbl> <dbl> <int>
1 tubers       0      4  4.72  2.28   3      6    13    478
2 weight       0     61 72.8  53.6   26.1  109.  323.   478
\end{verbatim}

bzw. ohne \texttt{plant}

\begin{Shaded}
\begin{Highlighting}[]
\NormalTok{pot }\SpecialCharTok{\%\textgreater{}\%} 
  \FunctionTok{select}\NormalTok{(}\SpecialCharTok{!}\NormalTok{plant)}\SpecialCharTok{\%\textgreater{}\%} 
  \FunctionTok{summarise}\NormalTok{(}\FunctionTok{across}\NormalTok{(}\FunctionTok{where}\NormalTok{(is.numeric), }\AttributeTok{.fns =} 
                     \FunctionTok{list}\NormalTok{(}\AttributeTok{min =}\NormalTok{ min,}
                          \AttributeTok{median =}\NormalTok{ median,}
                          \AttributeTok{mean =}\NormalTok{ mean,}
                          \AttributeTok{stdev =}\NormalTok{ sd,}
                          \AttributeTok{q25 =} \SpecialCharTok{\textasciitilde{}}\FunctionTok{quantile}\NormalTok{(., }\FloatTok{0.25}\NormalTok{),}
                          \AttributeTok{q75 =} \SpecialCharTok{\textasciitilde{}}\FunctionTok{quantile}\NormalTok{(., }\FloatTok{0.75}\NormalTok{),}
                          \AttributeTok{max =}\NormalTok{ max, }
                          \AttributeTok{n=}\NormalTok{length))) }\SpecialCharTok{\%\textgreater{}\%}
  \FunctionTok{pivot\_longer}\NormalTok{(}\FunctionTok{everything}\NormalTok{(), }\AttributeTok{names\_sep=}\StringTok{\textquotesingle{}\_\textquotesingle{}}\NormalTok{, }\AttributeTok{names\_to=}\FunctionTok{c}\NormalTok{(}\StringTok{\textquotesingle{}variable\textquotesingle{}}\NormalTok{, }\StringTok{\textquotesingle{}.value\textquotesingle{}}\NormalTok{))}
\end{Highlighting}
\end{Shaded}

\begin{verbatim}
# A tibble: 3 x 9
  variable    min median   mean  stdev   q25   q75   max     n
  <chr>     <dbl>  <dbl>  <dbl>  <dbl> <dbl> <dbl> <dbl> <int>
1 tubers        0      4  4.72   2.28    3      6    13    478
2 weight        0     61 72.8   53.6    26.1  109.  323.   478
3 infection     0      0  0.335  0.472   0      1     1    478
\end{verbatim}

für unterschiedliche Methoden:

\begin{Shaded}
\begin{Highlighting}[]
\NormalTok{pot }\SpecialCharTok{\%\textgreater{}\%}  \FunctionTok{group\_by}\NormalTok{(method) }\SpecialCharTok{\%\textgreater{}\%}
  \FunctionTok{summarise}\NormalTok{(}\FunctionTok{across}\NormalTok{(}\FunctionTok{c}\NormalTok{(}\StringTok{"tubers"}\NormalTok{, }\StringTok{"weight"}\NormalTok{), }\AttributeTok{.fns =} 
                     \FunctionTok{list}\NormalTok{(}\AttributeTok{min =}\NormalTok{ min,}
                          \AttributeTok{median =}\NormalTok{ median,}
                          \AttributeTok{mean =}\NormalTok{ mean,}
                          \AttributeTok{stdev =}\NormalTok{ sd,}
                          \AttributeTok{q25 =} \SpecialCharTok{\textasciitilde{}}\FunctionTok{quantile}\NormalTok{(., }\FloatTok{0.25}\NormalTok{),}
                          \AttributeTok{q75 =} \SpecialCharTok{\textasciitilde{}}\FunctionTok{quantile}\NormalTok{(., }\FloatTok{0.75}\NormalTok{),}
                          \AttributeTok{max =}\NormalTok{ max, }
                          \AttributeTok{n=}\NormalTok{length))) }\SpecialCharTok{\%\textgreater{}\%}
  \FunctionTok{pivot\_longer}\NormalTok{(}\AttributeTok{cols =} \SpecialCharTok{{-}}\NormalTok{method,  }\AttributeTok{names\_sep=}\StringTok{\textquotesingle{}\_\textquotesingle{}}\NormalTok{, }\AttributeTok{names\_to=}\FunctionTok{c}\NormalTok{(}\StringTok{\textquotesingle{}variable\textquotesingle{}}\NormalTok{, }\StringTok{\textquotesingle{}.value\textquotesingle{}}\NormalTok{)) }\SpecialCharTok{\%\textgreater{}\%} 
  \FunctionTok{arrange}\NormalTok{(variable)}\SpecialCharTok{\%\textgreater{}\%} 
  \FunctionTok{relocate}\NormalTok{(variable)}
\end{Highlighting}
\end{Shaded}

\begin{verbatim}
# A tibble: 8 x 10
  variable method       min median   mean stdev   q25   q75   max     n
  <chr>    <chr>      <dbl>  <dbl>  <dbl> <dbl> <dbl> <dbl> <dbl> <int>
1 tubers   aeroponic   0       5     4.76  2.29  3.75   6    13     120
2 tubers   bed         1       5     5.55  2.37  4      7    13     120
3 tubers   hydroponic  1       3     3.15  1.67  2      4     8     118
4 tubers   pot         2       5     5.4   1.89  4      7    10     120
5 weight   aeroponic   0      62.1  74.7  57.7  36.2   99.2 323.    120
6 weight   bed        13.2    79.6  84.5  43.6  50.3  118.  227.    120
7 weight   hydroponic  2.95   20.0  21.8  12.1  14.1   25.4  58.7   118
8 weight   pot        11.3   111.  109.   45.9  72.7  143.  248.    120
\end{verbatim}



\end{document}
